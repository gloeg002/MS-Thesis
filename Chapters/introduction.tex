%==============================================================================
%  introduction.tex: INTRODUCE THE TOPIC
%==============================================================================

% ----------------- DECLARE CHAPTER TITLE ----------------- %
\chapter{Introduction}
\label{intro_chapter}
%------------------------------------------------------------------------------

%%%%%%%%%%%%%%%%%%%%%%%%%%%%%%
% BACKGROUND
%%%%%%%%%%%%%%%%%%%%%%%%%%%%%
\section{Background}

%-----------------------------------------------------------------------------
% INTERNAL WAVES
%------------------------------------------------------------------
\subsection{Internal Waves}
\begin{comment}
discuss first observations of internal waves and what an internal wave actually is. 
\end{comment}

The Norweigen explorer Fridtjof Nansen made the first observations of
internal waves in 1893 while exploring the arctic. %\cite{briscoe_1975}. 
Nansen's ship, the Fram, had a reduced motion when sailing atop fresh water. 
The phenomenon, which Nansen called 'dead water', was a result of a internal wave at the interface between
the low density fresh water and high density saline water.  The details of the phenomenon were later explained by Ekman \citep{ekman1904dead}. 

Internal waves occur at pycnoclines in the fluid, because the density difference leads to a gravitational or hydrostatic pressure restoring force
 if the fluid is displaced vertically. Therefore, a more descriptive term is an internal gravity wave.  
 Two important pycnoclines in the ocean are the halocline (salinity gradient) and the thermocline (temperature gradient), 
 the latter being more important in lacustrine environments. The frequency  of internal waves are typically lower than that of surface gravity waves, 
 because density differences within a fluid are usually very small. The wavelength and period of internal waves span a wide range, centimeters to 
 kilometers and seconds to hours respectively. The amplitude of internal waves can be orders of magnitude larger than surface gravity waves. 

There is a whole classification of internal waves depending on the fluid stratification, wave amplitude, and generation mechanism. Interfacial
waves propagate along the interface separating fluids of different densities, large amplitude interfacial waves are termed solitary waves (wave of translation) and 
lee waves are generated by flow over topography. \footnote{An example of a lee wave is the Sierra wave, which is the result of wind being pushed up the Sierra Nevada mountain range 
and condensing forming a lenticular cloud downwind.} This thesis will focus on near-inertial waves, which are waves
whose dynamics are influenced by the Coriolis effect when the wave frequency is comparable to the Earth's rotational frequency. 

Two types of basin scale internal waves are typically observed in large lakes, namely Kelvin waves and Poincar\'{e} waves. Kelvin waves are shore hugging
internal waves which have the most influence between the coast and a distance of one internal Rossby radius.  These waves propagate around the the basin of the lake
and produce no flow normal to the shore. The amplitude decays off shore exponentially fast, $exp(\frac{-fx}{c})$, where $f$ is the Coriolis frequency, $x$ is the distance from shore, and $c$ is the internal wave speed \citep{mortimer_1974}. Poincar\'{e} waves on the other hand produce motion
across the entire basin and have little effect near shore \citep{antenucci2001energetics}, this type of wave will be the focus of this thesis.

%When the earth's rotation starts to become important for low frequency waves the wave is referred to as Poincare waves. 
%The result of earth's rotation is a clockwise rotating velocity signal in the northern hemisphere and a counterclockwise rotating 
%velocity signal in the southern hemisphere. 

%-----------------------------------------------------------------------------
% NEAR-INERTIAL WAVES
%-----------------------------------------------------------------------------
%NOTES : Lake Geneva \citep{bauer_etal_1981}; 
%		  Lake Iseo \cite{valerio_2012}. 
%  		  Lake Zurich \cite{horn_1986}

\subsection{Near-Inertial Waves}
Near-inertial waves (NIWs) are propagating Poincar\'{e} waves, in other words, the frequency is super-inertial but close to the inertial frequency. 
Large wavelength NIWs are commonplace in the ocean and atmosphere.  
\citet{hongbo_et_al_1995} used the phase difference between moored sites in the ocean to estimate the wavelength of NIWs to be between 80 km - 180 km. 
Wavelengths on this scale are not permissible in large lakes, such as Lake Superior, since it is comparable to the basin scale. Although wavelengths of this order
are not not form at the surface in Lake Superior, NIWS can propagate along sharp density differences in the lake, such as at the thermocline. The (internal) Rossby radius of deformation, 
defined as $R=\frac{c}{f_o}$,  where $c$ is the (internal) wave speed and $f_o$ is the local Coriolis frequency, quantifies the length scale at which rotation becomes important in a basin. 
 For large lakes the Rossby radius at the surface is much larger than the basin scale, therefore
NIWs will be weakly observed at the surface in surface water level records. The internal Rossby radius for Lake Superior is on the order of 10 km, which implies that NIWs can be observed 
within the lake where pycnoclines are the undulating surface. 

Throughout the 1960s and 1970s extensive observations of NIWs were made throughout the Laurentian Great Lakes using moored platforms to
collect data on the thermal and current structure of various lakes. Observations of currents were made using vector averaging current meters, such as a geodyne current meter, 
which correlates the number of  revolutions of a rotor in a sampling period to speed.  The direction of the rotor at the end of the sampling
period is taken to be the direction of the current. These types of measurements are crude compared with observations from an acoustic doppler current profiler, or ADCP, \citep{bugnon_1991}.  
An ADCP sends out a pulse of sound which reflects off passively moving particles in the water. The range of the object can be calculated by $R=\frac{1}{2}tc$, where $t$ is the 
round trip time of the signal and $c$ is the speed of sound in water ($\approx$ 1500 ms$^{-1}$). The Doppler shifted frequency yields 
information about the velocity of the moving particle, $f_d=\frac{2v}{c}f_o$, where $f_d$ is the Doppler shift in the original frequency $f_o$ and $v$ is the velocity of the source. 
The ADCP's transducers are arranged in a Janus pattern, which allows the current vector to broken down into zonal, meridional, and vertical 
components. 

During the International Field Year for the Great Lakes (IFYGL) a comprehensive interdisciplinary study was done on Lake Ontario \citep{ifygl}. A staggered grid of moorings were
placed throughout Lake Ontario (average spacing of 15 km), most stations recorded current at 15 and 30 m depth and temperature at 0,10,15,30, and 50 m. However, a few
other stations sampled at various other depths. The July 1972 thermal structure showed spectral energy at the inertial frequency (0.057 cycle hr$^{-1}$ ) to be greatest at the thermocline 
and the spatially averaged current to be greatest above the thermocline  \citep{pickett_1975}. The same observation was also observed in Lake Erie \citep{boyce_etal_1987}. 
Observations of winter (December 1972 - March 1973) currents in Lake Ontario  at 15 and 75 m showed inertial kinetic energy, calculated using band-passed filtered current velocities, 
to be greater during period of stratification \citep{marmorino_1978}. 
A similar study with 21 stations in Lake Huron showed that inertial motion was prominent during the stratified period of June-August 1966 and that this motion was clearly defined during periods of strong stratification  \citep{sloss_1976}. \citet{marmorino_1978}  also showed the band-passed current time series at the 15 and 75 m depth to be about 180 degrees out of phase, which shows the baroclinic nature of inertial currents. \citet{blanton_1974} studied the season variation of currents in 1970 throughout the water column in Lake Ontario and showed that inertial currents in the spring (May 16 - June 2), summer ( July 4 - July 18),  and fall (October 1 - October 19) accounted for about 20\%, 50\%, and 10\% of the variance respectively at a site 16 km offshore. \citet{blanton_1974}  also showed that inertial energy decreases closer to shore. 

Recent observations in Lake Michigan and Lake Superior have corroborated the findings of these earlier studies. \citet{choi_2012} showed that NIWs were prominent during the stratified season and accounted for about 80\% of the surface observations at an open water site in Lake Michigan. Observations in Lake Superior from 2008-2011 show that near-inertial response is related to the strength of stratification and that motion above and below the thermocline are roughly 180 degrees out of phase \citep{austin_2013}. \citet{austin_2013} analyzed
NIWs as propagating plane waves and made estimates of the wave climate in Lake Superior. He found the wavelength to vary between 30-60 km and the group speed (speed energy is transferred) to be around 22 cm s$^{-1}$. An interesting finding made by \citet{austin_2013} was that the direction of these waves slowly veers counterclockwise with a period of about 1 month,  which he noted has the same direction and period as the lowest order internal Kelvin wave.

% -- AUSTIN WIND PLAYS IMPORTANT ROLE
\begin{comment}
\citet{austin_2013} also makes the point that the strength of the stratification is not predictive of the amplitude of inertial oscillations and suggests the wind field plays a more important role.  
\end{comment}


The spatial structure of NIWs is not well characterized and all previous studies have approached the problem with modal analysis. 
Early work in this area was done by \citet{schwab_1977} in a flat bottom Lake Ontario model. He used a numerical procedure to calculate the amplitude and phase distribution of three  
Poincar\'{e} type modes. \citet{schwab_1977} calculated the three lowest Poincar\'{e} modes to have period of 16.8, 16.7, and 16.6 hours. A peak centered at these periods was observed 
by \citet{pickett_1975} in the average power spectra of temperature. However,  \citet{pickett_1975} observed spectra of current components to be closer to the inertial period (17.35 hours). 
\citet{schwab_1977} suggested it would be hard to separate these modes by standard methods and that the observed temperature spectra most likely contained energy from a few modes.
 Numerical modeling in Lake Michigan by  \citet{ahmed2013spatial} show the spatial structure of dominant internal modes. 
Analysis of isothermal displacements revealed three nodes in Lake Michigan with clockwise phase propagation around each node. 
%The largest inertial currents occur off shore and amplification of currents and thermocline undulations  have been observed over sloping bathymetry, possible due to wave reflection off the 
bottom \citep{gomez_2006}. 

The partition of energy in NIWs has been of particular interest in Lake Kinneret (Sea of Galilee)  \citep{antenucci2001energetics}.  Potential energy is
carried in undulations of pycnoclines, while kinetic energy is carried in the currents. Large currents are observed in the top mixed layer while weaker currents are
observed below the thermocline.  This is because the wind stress directly acts on the mixed layer producing larger currents. 
When the surface water subsides, such as at a coast, the thermocline will be depressed
generating a weak return flow in the bottom layer. The resulting oscillations at the thermocline are analogous to surface gravity waves. The ratio of potential
to kinetic energy in inertial waves is dependent on the Burger number defined as $Bu=(\frac{R_d}{L})^2$, where $R_d$ is the Rossby radius of deformation and $L$ is the
length scale of the basin \citep{antenucci2001energetics}. Antenucci and Imberger note that high Burger number basins have equal partition between potential and kinetic energy while 
low Burger number basins, such as Lake Superior, have an energy partition dominated by kinetic energy. 

% -- DISCUSS OBSERVATIONS IN OCEANIC ENVIRONEMNT 
%Substantial observations have been made in the ocean, quite recently by \citet{kim_2013} off the coast of Oregon. 

% -- NOTES
%First conceptual model for study basin scale internal waves (Mortimer 1952)
%Burger number : ratio of  time the basin takes to rotate to the period of the internal wave mode oscillation

% -- EFFECTS OF BATHYMETRY ON BOTTOM
%Nonuniform bottom bathymetry may dramatically modify the spatial structure of natural modes, by amplifying the wave velocities over sloping
%bottoms \cite{fricker_2000} \cite{gomez_2006} and magnifying velocities and wave amplitudes over local bathymetric features. Besides the
%bathymetry, embayments may also modify the Poincare type structure, introducing accompanying cyclonic cells \cite{gomez_2006}; 
%further, they may generate subbasin oscillations and increase the integrated kinetic energy oscillations where the boundary is convex \cite{rueda_2003}

%Current and temperature spectra intensification over sloping bottoms \cite{eriksen_1982}. 

% -- NOTES
%Brief discussion of what a near-inertial wave is (would go more in depth in mathematical background chapter)
%Discussion of past observations and modeling of NIOs in lakes and the ocean. 
%Maybe make a distinction between Kelvin waves and Poincar\'{e} waves. In addition, 
%I can make distinctions between small and large Burger number lakes. 
%Near-inertial waves are commonplace in the ocean and atmosphere. The waves are dominated by a nearly clockwise rotation of the current vector

%%%%%%%%%%%%%%%%%%%%%%%%%%%%
% MOTIVATION
%%%%%%%%%%%%%%%%%%%%%%%%%%%%%
\section{Motivation}

The motivation for this thesis stems from the proposed connection between near-inertial waves and the resuspension of nutrients during the stratified season. 
Near-inertial currents have been correlated with the thickness of the benthic nepheloid layer, suggesting these waves are important drivers resuspending 
nutrients and sediment \citep{hawley_2004}. Also, autonomous glider observations in Lake Superior show a link between stratification and backscatter 
of CDOM. \citep{austin_2013}. Therefore, understanding the spatial distribution of near-inertial energy could have ecological significance. In addition, these
waves need to be accurately modeled to correctly simulate the dynamics of nutrients, since plankton is observed to move with current and though 
the action of wave breaking nutrients can be moved into or out of the surface mixed layer. 

Very little is known about the spatial structure of near-inertial waves. Studies that have been done analyzed the waves
as basin scale normal modes \citep{schwab_1977, rueda_2003, gomez_2006, ahmed2013spatial}. Schwab (1977) numerically calculated
some of the inertial modes for Lake Ontario while Gomez-Giraldo et al. (2006) and Ahmed et al. (2013) used numerical simulations to 
study the spatial structure of these waves. Modeling studies have shown that the spatial structure of these waves is too complex to be predicted
by internal wave modes and that bathymetry plays an important role in the spatial distribution of energy \citep{rueda_2003}. Because of the complex nature 
of modal analysis for irregular basins the approach taken in this thesis will follow Austin (2013) and analyze NIWs in terms of the wave climate. 

The deep chlorophyll maximum, which tends to coincide with the thermocline, is commonplace in the ocean \citep{cullen1982deep} and 
during the stratified season in Lake Superior \citep{barbiero_2004}. Internal waves are characterized by undulations of the thermocline and 
therefore can potentially move phytoplankton rich water downward or, though wave breaking, bring nutrient rich water into the surface 
mixed layer \citep{haury1979tidally} \citep{haury1983tidally}. 

The other motivating factor for this study is to better understand the relationship between wind field and near-inertial wave energy. 
Using one dimensional models \citet{pollard_1969,pollard_1970} were able to show a few important relationships. The first relationship
is that a wind stress acting for less than one inertial period most effectively puts energy into inertial motions \citep{pollard_1969}. Imagine a wind
stress acting for one inertial period, the amount of momentum put into the inertial motion during the first half will be taken out during the second half. Therefore, 
a wind stress acting for half an inertial period effectively puts the most energy into inertial motion. Secondly, a wind stress rotating
clockwise in the northern hemisphere (counter-clockwise in the southern hemisphere) will continuously put energy into inertial motions \citep{pollard_1970}.
For example, if the wind stress were rotating at the inertial frequency then the direction of the wind stress and direction of the current would always be in 
the same direction and therefore momentum could always be put into the motion. This has been shown in observations as well,  
Boyce et al.(1987) presented observational evidence that large near-inertial events are associated more with clockwise rotating
 wind vectors than the amplitude of the wind stress \citep{boyce_etal_1987}.Todays models do not have the predictive power to determine when and 
 where large inertial events will occur and it is hoped that someday these models can make accurate predictions of when and where large near-inertial events will occur. 

% vertical shear of near-inertial currents contributes to vertical mixing and entrainment (ivey patterson 1984)
% bottom currents at inertial frequency supplyenergy for hypolimnetic mixied layer unique to that basin (ivey boyce 1982).


%%%%%%%%%%%%%%%%%%%%%%%%%%%%
% LAKE SUPERIOR
%%%%%%%%%%%%%%%%%%%%%%%%%%%%%
\section{Lake Superior}

Lake Superior formed approximately 11,000 years ago as the Laurentide ice 
sheet retreated northeastward across what is now Canada \citep{thomas_etal_1978}. Lake Superior is part 
of the Laurentian Great Lakes System, which also includes Lake Huron; Lake Ontario;
Lake Michigan; and Lake Erie. It is the largest lake of the Laurentian Great Lakes by surface area 
(82,100 km$^2$) and volume (12,100 km$^3$). 

Worldwide, Lake Superior is the largest lake by surface area and third largest by volume. The
lake is also an important water source and holds approximately 20\% of the world's freshwater supply. 
Lake Superior has a length of 560km, and breadth of 260km. The retention time of the lake is about 
200 years. The lake is home to 38 native species of fish, including ciscoes, whitefish, and trout. 

Lake Superior can be divided into two deep basins separated by a relatively shallow ridge off the Keweenaw peninsula. 
The average depth of the lake is about 150 m and has a maximum depth of 406 m. 


% -- LAKE SUPERIOR BATHYMETRY
% \FIGURE{size}{figure name in Figures dir.}{caption}{label}
\FIGURE{2.5}{bathymetry.pdf}{Lake Superior Bathymetry}{Bathymetry of Lake Superior. Mean depth of 150 m and max depth of 406 m. 
	A shallow ridge extending off the Keweenaw Peninsula separates the lake into two deep basins . 
	(source : NOAA)}


%--------------------------------------------------------------------------------
% THERMAL STRUCTURE
%--------------------------------------------------------------------------------
\subsection{Thermal Structure}
Freshwater lakes stratify because the temperature of maximum density (T$_{md}$=3.98$^\circ$C) is greater than the freezing point (T$_{f}$=0$^\circ C$) 
at zero gauge pressure.  $T_{md}$ decreases as pressure increases therefore,  $T_{md}$ is lower at greater depth. A plot of freshwater density vs temperature is given below. 

% -- DENSITY PLOT
% \FIGURE{size}{figure name in Figures dir.}{caption}{label}
\FIGURE{3}{density.pdf}{Freshwater Density}{Fresh water density as a function of temperature at zero gauge pressure \citep{chen_millero_1986}. 
%Note the small density difference between   0$^\circ C$ and 4$^\circ C$ and the relatively large density difference between 0$^\circ C$ and 20$^\circ C$. 
 \newline }


Lets assume the entire water column is  $T_{md}$, surface heat fluxes will then cause the surface of the lake to warm or cool away from $T_{md}$ and this less dense water will float atop 
the more dense water below. Water nearest to the bottom of the lake will remain very close to $T_{md}$ throughout the entire year. As more and more heat enters the surface of the lake 
the water column will separate into two distinct layers. The epilimnion, or surface mixed layer, and the more stagnant hyplimnion below. At the intersection between these two layers is the 
metalimnion, which is characterized by the thermocline or sharp temperature/density gradient. As the epilimnion heats up the density difference between the two layers increases 
leading to a steeper and more stable thermocline. 

Just after stratification the hypolimnion is rich in oxygen from the spring mixing. Since the epilimnion acts as a
barrier to the hypolimnion,  deep water is cut off from oxygen exchange with the atmosphere and therefore this region can become 
anoxic as oxygen is consumed by bottom dwellers. During stratification the lake can partially mix as cold fronts pass over with high winds. 

The beginning of stratification varies around the lake, typically late June near the shore and mid-July in open water. November is typically the end of the stratified season. 
With global temperature increasing it is likely to expect stratification to start early and have a prolonged period of stratification. Negative stratification occurs in the winter. 
The bottom of the lake remains close the temperature of maximum density while surface water is colder and less dense, hence this water floats atop the dense deep water. 
However, water freezes at $0^{\circ}C$ so the density difference between these two layers is very small, but is enough to change the dynamics of the water column. 
During the summer stratified season the density difference between the surface and bottom is close to $20*10^{-4} \text{g\ cm}^{-3}$ and during the winter when the lake is inversely stratified this
density difference is around $5*10^{-4} \text{g\ cm}^{-3}$.  Lake Superior is a dimicitic lake and therefore stratifies twice pear year, once in summer and once in winter. The thermal cycle in Lake Superior is provided below at the core mooring sites. 

% -- LS THERMAL STRUCTURE
\FIGURE{5}{LS_thermistor.pdf}{Lake Superior Thermal Structure}{Thermal cycle of Lake Superior. 
The solid black line shows the epilimnion temperature (measurement taken either 1 or 10 m from the surface) and the dotted red line is the thermistor 250 m from the surface. \newline }

The heat content of Lake Superior can be broken down into the "spring heat income" and  the "annual heat income", where are described by \citep{bennett_1978}.  The spring heat income is the
heat gained by the water column moving from a state of minimum winter heat content to the time the water column begins to stratify in the summer. The annual heat income is the difference 
between the minimum the maximum heat content. For Lake Superior the minimum and maximum heat contents around April 1st and October 1st respectively. The spring heat income for 
Lake Superior is about 1.5 x 10$^{9}$ $Jm^{-2}$ and the annual heat income is about 2.4 x 10$^{9}$ $Jm^{-2}$ \citep{titze_thesis}. Mid-lake upwelling of warm water in the winter tends to keep the central area free of ice and enhance heat loss \citep{bennett_1978}. 

%====================================================================================


