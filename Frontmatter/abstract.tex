%%%%%%%%%%%%%%%%%%%%%%%%%%%%%%%%%%%%%%%%%%%%%%%%%%%%%%%%%%%%%%%
% abstract.tex: Abstract
%%%%%%%%%%%%%%%%%%%%%%%%%%%%%%%%%%%%%%%%%%%%%%%%%%%%%%%%%%%%%%%

Numerical modeling and observational data of temperature and currents were used to study the spatial 
structure and temporal evolution of near-inertial waves (NIWs). A hierarchy of models were ran using ideal 
wind stress as well as spatially and temporally varying wind stress derived from NARR wind speeds. 
Results indicate that a wind stress lasting for half an inertial period is most efficient at putting kinetic energy into
the system at the Coriolis frequency.  Although ideal forcing puts energy in at the Coriolis frequency, it puts about four
orders of magnitude less energy in at other frequencies. High energy away from the Coriolis frequency in observations is likely
the result of vagaries in the wind field. A model of Lake Superior was ran for the period of July, 1 2011 to 
September, 19 2011. Lake Superior was modeled as a closed basin and radiative forcing was neglected in the model in order
to solely look at the effects of wind stress on inertial currents. Model output was compared with ADCP observations from 
three mooring sites in Lake Superior. 

%%%%%%%%%%%%%%%%%%%%%%%%%%%%%%%%%%%%%%%%%%%%%%%%%%%%%%%%%%%%%%%